%%% noise.tex ---

\documentclass[twoside,10pt]{article}

% \usepackage{epsfig}
\usepackage{color}
\usepackage{fancyhdr}
\usepackage{rotating}
\usepackage{multirow}
\usepackage{graphicx}
\usepackage{amssymb}
\usepackage{nicefrac}
\usepackage{cool}
\usepackage{mathrsfs}
\usepackage{units}
\usepackage{amsmath}
\usepackage{subfig}
\Style{DSymb={\mathrm d},IntegrateDifferentialDSymb=\text{d}}

% define formatting
% A4 paper is 210 x 297 mm
% pt  point         (1 in = 72.27 pt)
% pc  pica          (1 pc = 12 pt)
% in  inch          (1 in = 25.4 mm)
% bp  big point     (1 in = 72 bp)
% cm  centimetre    (1 cm = 10 mm)
% mm  millimetre
% dd  didot point   (1157 dd = 1238 pt)
% cc  cicero        (1 cc = 12 dd)
% sp  scaled point  (65536 sp = 1 pt)
% em  the width of the letter M in the current font
% ex  the height of the letter x in the current font

\setlength{\textheight}{11.in}
\setlength{\textwidth}{8.5in}

\setlength{\topmargin}{0.in}
\setlength{\headheight}{\baselineskip}
\setlength{\headsep}{2\baselineskip}

\addtolength{\textheight}{-\headheight}
\addtolength{\textheight}{-\headsep}
\addtolength{\textheight}{-2.in}
\addtolength{\textwidth}{-2.in}


\setlength{\oddsidemargin}{0.in}
\setlength{\evensidemargin}{0.in}

\setlength{\parindent}{0.em}
\setlength{\parskip}{0.75\baselineskip}

\def\boxwidth{\textwidth}
\def\doctitle{Sky map instrument noise}         % Document title defined here
\def\myname{Justin Lazear}      % Primary author defined here
\def\myemail{\texttt{jlazear@pha.jhu.edu}} % Primary author email
\def\piper{\textsc{Piper}}
\def\tablefont{\footnotesize}

% \renewcommand{\thefootnote}{\fnsymbol{footnote}}

%% \pagestyle{fancy}
%% \fancyhead{}
%% \fancyhead[LE,RO]{{\it {\large \thepage}}}
%% \fancyhead[RE]{{\it {\large \myname{}}}}
%% \fancyhead[LO]{{\it {\large \doctitle{}}}}
%% \fancyfoot{}

%%%%%%%%%%%%%%%%%%%%%%%%%%%%%%%%%%%%%%%%%%%%%%%%%%%%%%%%%%%%%%%%%%%%%%

% Define some math macros for convenience

% Note that most necessarily math macros are defined by the cool
% package, so this section is sparse.

\DeclareMathOperator{\rect}{rect}
\DeclareMathOperator{\sinc}{sinc}

\newcommand{\ve}[1]{\mathbf{#1}}
\newcommand{\Avg}[1]{\left< #1 \right>}
\newcommand{\pp}[1]{\left( #1 \right)}
\newcommand{\pb}[1]{\left[ #1 \right]}
\newcommand{\pc}[1]{\left{ #1 \right} }
\newcommand{\dd}[0]{\mathrm{d}}
\newcommand{\NEP}[0]{\mathrm{NEP}}
\newcommand{\NET}[0]{\mathrm{NET}}
\newcommand{\NEQ}[0]{\mathrm{NEQ}}
\newcommand{\NEU}[0]{\mathrm{NEU}}
\newcommand{\NEV}[0]{\mathrm{NEV}}
\newcommand{\kB}[0]{k_\mathrm{B}}
\newcommand{\KCMB}[0]{K_\mathrm{CMB}}

%%%%%%%%%%%%%%%%%%%%%%%%%%%%%%%%%%%%%%%%%%%%%%%%%%%%%%%%%%%%%%%%%%%%%%

\title{\doctitle}
\author{\myname}

\begin{document}
\maketitle

\section{Sky map instrument noise}
\label{sec:noise}

Let us estimate here the noise in each sky (HealPIX) pixel coming from the
intrinsic noise in each detector pixel.

\section{NEP}
\label{sec:nep}

Detector noise is most conveniently quoted in noise-equivalent power (NEP),
which is defined as the amount of optical power that must be incident such
that the signal-to-noise ratio (S/N) is 1. It is a measure of noise in optical
power units. Note that NEP may be referenced at various different points in
the instrument, e.g. inside the detector, incident on the detector, incident
at the aperture of the telescope. The noise (N) component is intrinsic to the
detector, and so is always referenced to inside the detector. Thus, the S/N =
1 condition must always be enforced inside the detector. For NEPs referenced
at some other point, the signal (S) component must first be transferred to
inside the detector, and then compared. Explicitly,

\begin{description}
    \item[NEP (strict definition)] \hfill \\
    The amount of optical signal power incident at the reference
    point that creates a signal-to-noise ratio of 1 (S/N = 1) as measured
    inside of the detector.
\end{description}

From Richards\cite{richards_bolometers_1994}, we note that the noise in a
bolometer is

\begin{equation}
    \label{eq:noisepower}
    \frac{P_N^2}{B} = 2 \int\dd\nu h^2\nu^2 2N(n + n^2) = 2 \int\dd\nu P_\nu h\nu + \int\dd\nu P_\nu^2 \frac{c^2}{A\Omega \nu^2} \qquad\qquad \left[ \frac{\unit{W^2}}{\unit{Hz}} \right]
\end{equation}

where the integral is taken over the passband and the first term on the right
hand side corresponds to $n$ and the second term corresponds to $n^2$. $B$ is
the detector bandwidth, $N$ is the number of modes ($N = A\Omega/\lambda^2$,
from the Antenna Equation), $n$ is the number of photons per mode, the 2 in
front of the first integral comes from the conversion between integration time
and bandwidth (see below and Appendix~\ref{sec:bw}), and the 2 inside the
integral comes from the 2 polarization states per photon. The energy per
photon is $h\nu$. The term $n + n^2$ is the thermal expectation value for the
variation in the number of photons per mode, $\Avg{(\Delta n)^2} = n + n^2$.

We note that the number of photons per mode is\cite{kittel_thermal_1980}

\begin{equation*}
    n = \frac{1}{\exp{\left( \frac{h\nu}{\kB T} \right)} - 1}
\end{equation*}

where $T$ references the signal (sky) temperature, which is the temperature of
the CMB\cite{fixsen_temperature_2009} in this case, $T = \unit[2.726]{K}$. For
the 4 \piper\ frequency bands (200, 270, 350, 600 GHz), this gives photon
numbers $n$

\begin{align*}
    n = 3\times 10^{-2} & \text{ for } \nu = \unit[200]{GHz} \\
    9\times 10^{-3} & \text{ for } \nu = \unit[270]{GHz} \\
    2\times 10^{-3} & \text{ for } \nu = \unit[350]{GHz} \\
    3\times 10^{-5} & \text{ for } \nu = \unit[600]{GHz}
\end{align*}

We note that $n \ll 1$ for all frequency bands, so $n^2 \ll n$, and we may
ignore the second term in Eq.~\eqref{eq:noisepower}. So we use for the noise
power,

\begin{equation}
    \label{eq:noisepower2}
    \frac{P_N^2}{B} = 2 \int\dd\nu P_\nu h\nu \qquad\qquad \left[ \frac{\unit{W^2}}{\unit{Hz}} \right]
\end{equation}

where $P_\nu$ is the power spectral density.

Let us compute the NEP inside the detector using this expression. The NEP is
defined as the amount of power required to give a signal to noise ratio of 1
when there is 1 Hz of noise bandwidth. Note that the signal bandwidth is
already implicitly integrated out of this expression. So we have,

\begin{align*}
    \frac{S}{N} = \frac{\NEP}{P_N(\unit[1]{Hz})} & = 1 \\
    \NEP & = P_N(\unit[1]{Hz}) \\
    \NEP^2 & = \left( \unit[1]{Hz} \right) \cdot \frac{P_N^2}{B}
\end{align*}

Note that in this scenario, NEP has units of W, which strictly matches the
definition given above. However, it is conventional to fold the 1 Hz of
noise bandwidth back into the definition of NEP to remind what the scaling
with bandwidth is and so that a factor of 1 Hz does not need to be carried
around,

\begin{equation*}
    \NEP^2_\mathrm{conventional} = \frac{\NEP^2}{\unit[1]{Hz}} = \frac{P_N^2}{B}
\end{equation*}

where the explicit definition of conventional NEP is

\begin{description}
    \item[NEP (conventional)] \hfill \\
    The amount of optical signal power incident at the reference
    point that creates a signal-to-noise ratio of 1 (S/N = 1) as measured
    inside of the detector, all divided by $\unit[1]{Hz^{1/2}}$.
\end{description}

Henceforth, we will use only the conventional NEP and drop the subscript.

Let us now consider a system with detector absorption
efficiency\footnote{Alternatively called detector absorptivity.} $\eta$ and
optical efficiency $\tau$, where $0 < \eta, \varepsilon, \tau \leq 1$. These
parameters make no difference to the NEP referenced inside the detector.

We will compute the NEP referenced to power incident on the detector. The
signal power incident to the detector will produce an amount of power inside
the detector that is reduced by the factor $\eta$. So the NEP condition is
then

\begin{equation*}
    \frac{\eta \cdot \NEP_D}{P_N} = 1
\end{equation*}

and we see that the NEP at the detector input is

\begin{equation}
    \NEP_D^2 = \frac{1}{\eta^2} \frac{P_N^2}{B} = \frac{2}{\eta^2} \int\dd\nu P_\nu h\nu
\end{equation}

Similarly, the NEP referenced to power incident on the primary of the
telescope is determined by noting that the power at the primary must pass
through the optical system, with losses according to the optical efficiency
$\tau$, and then be absorbed by the detector. So our NEP condition is

\begin{equation*}
    \frac{\eta \tau \cdot \NEP_P}{P_N} = 1
\end{equation*}

and the NEP at the telescope input is

\begin{equation}
    \NEP_P^2 = \frac{1}{\eta^2\tau^2} \frac{P_N^2}{B} = \frac{2}{\eta^2\tau^2} \int\dd\nu P_\nu h\nu
\end{equation}

We will assume that the NEP inside the detector is known, as it is typically
measured independently.

As a final note, most of the transformations in this section do not tell us
how to calculate the noise power of the bolometer, which is an intrinsic
aspect of the bolometer itself and should be computed from the perspective of
heat inside the detector. It is a property of the detector that is independent
of the optical loading, since optical power is converted to heat, and is
interchangeable with all other sources of heat in the detector (such as
electrical). Rather, what these transformations tell us is how to translate
the intrinsic noise in the detector (encoded as NEP inside the detector) to
reference points outside of the detector. Since the power is transported as
photons coming out of the detector (in the time-reversed sense), we must
understand the properties of the photons and how they transport power in order
to understand how the detector NEP translates to noise at other points. That
is the purpose of this section.

However, in addition to intrinsic detector noise, there is also intrinsic
noise in the signal from the sky in what is called photon noise. Photon noise
can be modeled by using $P_\nu = A\Omega \eta \tau B_\nu(T)$,

\begin{align}
    \NEP^2_\mathrm{photon} & = \left( \frac{2}{\unit[1]{Hz}} \right) A\Omega\,\eta\tau \left( \frac{2h^2}{c^2} \right) \int_{\Delta\nu}\ \dd\nu\ \frac{\nu^4}{\exp{\left( \frac{h\nu}{\kB T} \right)} - 1} & \left[ \frac{\unit{W^2}}{\unit{Hz}} \right] \\
    \NEP^2_\mathrm{photon} & = \left( \frac{2}{\unit[1]{Hz}} \right) A\Omega\,\eta\tau \frac{2h^2}{c^2} \left( \frac{\kB T}{h} \right)^5 \int_{x_1}^{x_2}\ \dd x\ \frac{x^4}{e^x - 1} & \left[ \frac{\unit{W^2}}{\unit{Hz}} \right]
\end{align}

where $x = \frac{h\nu}{\kB T}$. See below for details of the transformation.

Since it originates from the sky, it must be treated slightly differently. For
a single detector, there is no difference, but for many detectors, the optical
system can correlate pixels. This is impossible for noise intrinsic to the
detector, since each detector pixel is a unique independent device. Readout
noise could correlate different pixels to each other, but we do not consider
this case here.

\section{NET, NEQ, NEU, NEV}
\label{sec:net_neq_neu_nev}

The first quantity we will examine is the noise-equivalent temperature (NET).
This is the same as the NEP, except in temperature difference units
($\unit{\KCMB}$)\footnote{Also known as CMB temperature units. See
\texttt{docs/foregrounds} for a discussion of these units versus regular
thermodynamic temperature units.}. It is defined as the change in
thermodynamic temperature of a blackbody at the reference point that would
generate an amount of optical power in the detector such that the S/N is 1.
Note that this definition implicitly includes a fair amount of information
about the instrument, such as the etendue, the optical efficiency, and the
frequency band. This information is required since we must know how a change
in the temperature of the sky's photons propagate to the detector, and the
path of propagation is through the instrument. Also note that we must assume a
base sky temperature since the spectral distribution of the sky is
temperature-dependent.

The NET is defined relative to the NEP by

\begin{equation}
    \NET = \frac{\NEP}{\left( \D{P}{T} \right)} \qquad \qquad \left[ \unit{\KCMB/\sqrt{Hz}} \right]
\end{equation}

where $\D{P}{T}$ is the change in optical power incident on the detector due
to a change in sky temperature. This is straight-forward to compute if we
first find the power incident on the detector, which is

\begin{equation}
    P(T) = \int_{\Delta\nu} \dd\nu\ \eta(\nu) \tau(\nu) P_\nu(T) = A\Omega \int_{\Delta\nu} \dd\nu\ \eta(\nu) \tau(\nu) B_\nu(T) \qquad\qquad \left[ \unit{W} \right]
\end{equation}

where we have implicitly defined $P_\nu = \eta \tau A\Omega B_\nu(T)$ as the
spectral power density, $\Delta\nu$ is the passband and $B_\nu(T)$ is the
Planck distribution, which we note already includes both polarization modes,

\begin{equation*}
    B_\nu(T) = \frac{2h}{c^2} \frac{\nu^3}{\exp{\left( \frac{h\nu}{\kB T} \right)}}.
\end{equation*}

Then $\D{P}{T}$ is given by

\begin{equation*}
    \D{P}{T} = A\Omega \int_{\Delta\nu} \dd\nu\ \eta(\nu) \tau(\nu) \D{B_\nu(T)}{T}
\end{equation*}

The derivative of the Planck distribution can be shown to be

\begin{equation*}
    \D{B_\nu(T)}{T} = \frac{2h^2}{c^2\kB T^2} \frac{\nu^4 \exp\left( \frac{h\nu}{\kB T} \right)}{\left[ \exp\left( \frac{h\nu}{\kB T} \right) - 1 \right]^2}
\end{equation*}

so

\begin{equation}
    \D{P}{T} = A\Omega \int_{\Delta\nu} \dd\nu\ \eta(\nu) \frac{2h^2}{c^2\kB T^2} \frac{\nu^4 \exp\left( \frac{h\nu}{\kB T} \right)}{\left[ \exp\left( \frac{h\nu}{\kB T} \right) - 1 \right]^2} \qquad\qquad \left[ \frac{\unit{W}}{\unit{\KCMB}} \right]
\end{equation}

which may be put in the more numerically convenient form by using the
substitution $x = \frac{h}{\kB T} \nu$,

\begin{equation}
    \D{P}{T} = A\Omega \frac{2\kB}{c^2} \left( \frac{\kB T}{h} \right)^3 \int_{x_1}^{x_2} \dd x\ \eta(x)\tau(x) \frac{x^4 e^x}{\left( e^x - 1 \right)^2} \qquad\qquad \left[ \frac{\unit{W}}{\unit{\KCMB}} \right]
\end{equation}

The integral does not have a nice algebraic solution even if $\eta(x)\tau(x)$
is constant, but is amenable to quadrature for realistic bandpass functions
$\eta(x)$. We note, as discussed above, that the etendue ($A\Omega$), the
band-pass, the efficiency, and the sky temperature are involved in this
conversion factor.

The units of $P$ is \unit{W}, so $\left[\D{P}{T}\right] = \unit{\frac{W}{\KCMB}}$,
and

\begin{equation*}
    \left[ \NET \right] = \left[ \frac{\NEP}{\left( \D{P}{T} \right)} \right] = \unit{\frac{W}{\sqrt{Hz}}} \unit{\frac{K}{W}} = \unit{\frac{K}{\sqrt{Hz}}}
\end{equation*}

However the conventional unit for NET is $\unit{\KCMB \sqrt{s}}$, for which
the procedure to get the noise figure in K is to divide by the square root of
the integration time, i.e. more integration time results in less noise. This
is conceptually intuitive, since one would expect that the measurement from
each period of time would be independent, so the number of independent samples
would go like $f_\mathrm{sample} \cdot T_\mathrm{integration}$, and the noise
would go down by the square root of the number of independent samples.

We note that formally the units $\unit{\frac{\KCMB}{\sqrt{Hz}}}$ and
$\unit{\KCMB \sqrt{s}}$ are equivalent. However, the conversion between the
two is a bit more subtle. We wish to convert between units of bandwidth to
units of integration time, but due to the Nyquist sampling theorem, we must
integrate for 2 seconds to get 1 Hz of bandwidth (see Appendix~\ref{sec:bw}).
So the correct conversion factor is

\begin{equation*}
    1 = \frac{\unit[2]{s}}{\unit[1]{Hz^{-1}}}
\end{equation*}

Then the NET may be written in conventional units as

\begin{equation}
    \NET = \left( \sqrt{2} \frac{\unit{\sqrt{s}}}{\unit{Hz^{-1/2}}} \right) \frac{\NEP}{\left( \D{P}{T} \right)} \qquad \qquad \left[ \unit{\KCMB \sqrt{s}} \right]
\end{equation}

Next let us examine the noise-equivalent Q-parameter (NEQ). Again this is
simply a unit transformation, this time from temperature intensity units
($\unit{\KCMB}$) to polarization intensity units (also in $\unit{\KCMB}$). NEQ
is a noise-equivalent measure of the amount of noise in the measurement of the
Stokes Q parameter. There is also a noise-equivalent U-parameter (NEU), but it
is usually identical to NEQ, so typically only NEQ is listed. The conventional
units of NEQ are the same as NET, $\unit{\KCMB \sqrt{s}}$. There is also a
noise-equivalent V-parameter (NEV), which has the same units and a similar
interpretation. It is rarely listed, since the V-parameter is rarely of
interest.

To convert from NET to NEQ, we note that in order to measure the temperature
intensity, we need only to measure 1 number. However, to measure the
polarization, we must measure 3 numbers ($Q$, $U$, and $V$, or equivalently,
$E_x$, $E_y$, and $\phi$). Because of this, we must split our observation time
to measuring 3 different things, so for a single parameter, we get only a
fraction of the observation time. If the observation time is split between
$Q$, $U$, and $V$ according to the ratios $f_Q$, $f_U$, and $f_V$ (where $0
\leq f_X \leq 1$ and $f_Q + f_U + f_V = 1$), then we will reduce the
observation time for $Q$ according to $T_\mathrm{integration} \to f_Q
T_\mathrm{integration}$ (and similarly for U and V). Then since the number of
independent measurements is linear in the integration time, $N_\mathrm{obs} =
f_\mathrm{sample} T_\mathrm{integration}$, we get only a fraction of the
number of independent samples, $N_\mathrm{obs} \to f_Q N_\mathrm{obs}$. This
change results in a $1/\sqrt{f_Q}$ increase in noise, since noise goes like
$1/\sqrt{N_\mathrm{obs}}$. Thus, the generic conversions from NET to NEQ, NEU,
and NEV are

\begin{align*}
    \NEQ & = \left( \sqrt{\frac{2}{f_Q}} \frac{\unit{\sqrt{s}}}{\unit{Hz^{-1/2}}} \right) \frac{\NEP}{\left( \D{P}{T} \right)} \qquad \qquad \left[ \unit{\KCMB \sqrt{s}} \right] \\
    \NEU & = \left( \sqrt{\frac{2}{f_U}} \frac{\unit{\sqrt{s}}}{\unit{Hz^{-1/2}}} \right) \frac{\NEP}{\left( \D{P}{T} \right)} \qquad \qquad \left[ \unit{\KCMB \sqrt{s}} \right] \\
    \NEV & = \left( \sqrt{\frac{2}{f_V}} \frac{\unit{\sqrt{s}}}{\unit{Hz^{-1/2}}} \right) \frac{\NEP}{\left( \D{P}{T} \right)} \qquad \qquad \left[ \unit{\KCMB \sqrt{s}} \right]
\end{align*}

For \piper, the VPM modulation strategy results in a demodulation such that
$\sqrt{f_Q} = \sqrt{f_U} = \frac{1}{2} \cdot
\frac{4}{3}\sqrt{f_V}$.\footnote{See \piper\ proposal, if available to you.
Particularly, each telescope has 0.8 sensitivity to local Q and 0.6
sensitivity to V. Since sensitivity goes like $\sqrt{f}$, this gives us
$\sqrt{f_Q} = \frac{0.8}{0.6}\sqrt{f_V} = \frac{4}{3}\sqrt{f_V}$. But since
both telescopes measure V, and instrument Q and U are measured by only one
telescope each, the $V$ weight is doubled.} Combining this with the
normalization condition, $f_Q + f_U + f_V = 1$, we see that

\begin{equation}
    f_Q = f_U = \frac{4}{17} \simeq 0.2353 \qquad\text{and}\qquad f_V = \frac{9}{17} \simeq 0.5294.
\end{equation}

which gives us the \piper-specific conversions,

\begin{align}
    \NEQ = \NEU & = \left( \frac{\sqrt{34}}{2} \frac{\unit{\sqrt{s}}}{\unit{Hz^{-1/2}}} \right) \frac{\NEP}{\left( \D{P}{T} \right)} = \left( \frac{2.915 \unit{\sqrt{s}}}{\unit{Hz^{-1/2}}} \right) \frac{\NEP}{\left( \D{P}{T} \right)} \qquad \qquad \left[ \unit{\KCMB \sqrt{s}} \right] \\
    \NEV & = \left( \frac{\sqrt{34}}{3} \frac{\unit{\sqrt{s}}}{\unit{Hz^{-1/2}}} \right) \frac{\NEP}{\left( \D{P}{T} \right)} = \left( \frac{1.943 \unit{\sqrt{s}}}{\unit{Hz^{-1/2}}} \right) \frac{\NEP}{\left( \D{P}{T} \right)} \qquad \qquad \left[ \unit{\KCMB \sqrt{s}} \right]
\end{align}

Lastly, note that we have not specified the reference point for the NEP in any
of these calculations. All of these conversions are independent of the
efficiencies of the telescope and detector, so moving the reference point of
the NEP will change the reference point of the derived quantity (e.g. NEQ or
NEU) in the same way. Thus, the derived quantities (NEQ, NEU, NEV) have
reference points, and the reference point of the derived quantity is the same
as the NEP that was used to construct it. Transforming a derived quantity to
move the reference point is done in the same way that transforming the NEP is
done.

\section{Map Sensitivity}
\label{sec:sensitivity}

Now that we have in hand the NEQ, which describes the noise in a given
detector pixel and how it depends on integration time, we turn to estimating
the amount of noise in a given sky pixel. We have already accounted for the
noise properties of the detector (encoded in the NEP), so the frame of the
detectors is no longer convenient. We are really interested in the noise in
each sky pixel, not in the noise in each detector pixel, so we must project
the detector noise back onto the sky. Then we can accumulate the integration
time in each pixel and get the noise in each sky pixel.

Let us write the NEQ at the sky. We will assume that there is perfect transfer
from the sky to the telescope primary, i.e. there are no atmospheric effects
or CMB secondaries. These would factor in at the transfer from the primary to
the sky and would require some other efficiency parameter in addition to
$\eta$ and $\tau$.

\begin{equation*}
    \NEQ^2_P  = \left( \frac{2}{f_Q} \frac{\unit{s}}{\unit{Hz^{-1}}} \right) \frac{\NEP^2_P}{\left( \D{P}{T} \right)^2} = \left( \frac{2}{f_Q} \frac{\unit{s}}{\unit{Hz^{-1}}} \right) \frac{1}{\eta^2 \tau^2}\frac{\NEP^2}{\left( \D{P}{T} \right)^2}.
\end{equation*}

Over the course of an experiment, the detector spends some period of time
looking at each unit of solid angle on the sky. Supposing we have some kind of
idealized experiment that uniformly sampled a region of the sky with an
overlap factor $f_s$, then the integration time per unit solid angle is

\begin{equation}
    t = \frac{f_sT_e}{\Omega_e} = \frac{f_sT_e}{4\pi f_e}\qquad\qquad \left[\unit{\frac{s}{sr}}\right]
\end{equation}

To get a sense of the meaning of $f_s$, we can imagine two different
experiments with the same etendue $A\Omega$. The first experiment has a small
beam $\Omega_1$, and so in the experiment period $T_e$ can only cover the
experimental region $\Omega_e$ by raster scanning. Each beam spot gets an
integration time of only $T_e/(\Omega_e/\Omega_1)$. The second experiment has
a large beam $\Omega_2 > \Omega_1$, and so can cover the experimental region
more quickly, and so cover it more times in the given experimental period. In
particular, experiment 2 has an integration time of $T_e/(\Omega_e/\Omega_2)$.
If experiment 2 followed a similar raster scan strategy, it would complete
$f_s = \frac{T_e/(\Omega_e/\Omega_2)}{T_e/(\Omega_e/\Omega_1)} =
\Omega_2/\Omega_1$ scans in the time experiment 2 took to complete a single
scan.

Then as we noted before, the integration period and the number of independent
samples are related, so we should divide $\NEQ_P$ by $\sqrt{t}$ to get the
map sensitivity $m$,

\begin{align}
    m^2 & = \frac{\NEQ^2_P}{t} = \frac{1}{t}\left( \frac{2}{f_Q} \frac{\unit{s}}{\unit{Hz^{-1}}} \right) \frac{1}{\eta^2 \tau^2}\frac{\NEP^2}{\left( \D{P}{T} \right)^2} = \frac{4\pi f_e}{f_sT_e} \left( \frac{2}{f_Q} \frac{\unit{s}}{\unit{Hz^{-1}}} \right) \frac{1}{\eta^2 \tau^2}\frac{\NEP^2}{\left( \D{P}{T} \right)^2} \label{eq:m} \\
    m & = \frac{1}{\sqrt{t}} \left( \sqrt{\frac{2}{f_Q}} \frac{\unit{\sqrt{s}}}{\unit{Hz^{-1/2}}} \right) \frac{1}{\eta \tau} \frac{\NEP}{\left( \D{P}{T} \right)} \qquad\qquad \left[ \unit{\KCMB \sqrt{sr}} \right]
\end{align}

We note that $m$ has units of $[m] = \unit{\KCMB \sqrt{sr}}$. It represents
the amount of noise one would get in a region that subtends a particular solid
angle. Larger regions require more integration time, and so their noise is
reduced. Put another way, 1 second of integration time can measure a region of
a particular size (in solid angle) to some fixed amount of precision. To
measure a larger region (in solid angle), one must put together many of these
fixed size regions. Since each subregion is uncorrelated, this involves
combining the fluctuations of many random variables, which reduces the total
variance of the whole region.

As a final note, the map sensitivity can be trivially extended to include
non-uniform experiments. Simply let the integration time per solid angle
depend on direction, $t = t(\ve{n})$, then

\begin{equation}
    m^2(\ve{n}) = \frac{\NEQ^2_P}{t(\ve{n})} = \frac{1}{t(\ve{n})}\left( \frac{2}{f_Q} \frac{\unit{s}}{\unit{Hz^{-1}}} \right) \frac{1}{\eta^2 \tau^2}\frac{\NEP^2}{\left( \D{P}{T} \right)^2}.
\end{equation}

This expression contains the full directional dependence for our purposes,
because the NEQ depends solely on the properties telescope and detector. For a
space-based telescope, this is valid. For a balloon- or ground-based
experiment, it would be wise to include the effects of the atmosphere. Since
the atmosphere is most conveniently represented in the coordinate system of
the experiment and not in the CMB coordinate system, it is preferable to
include atmospheric effects in the integration time and to construct an
effective integration time, $t(\ve{n}) \to \tilde{t}(\ve{n})$. Using the
effective integration time $\tilde{t}$, the above equation would then encode
all of the directional dependencies.

\section{Multiple Detectors}
\label{sec:multiple_detectors}

Suppose instead of a single detector we have an array of detectors. We would
like to understand how to estimate the sensitivity of the full array from the
properties of the single pixel and the experiment properties. We recall that
the map sensitivity is computed from the NEQ via the integration time per
solid angle,

\begin{equation*}
    m^2 = \frac{\NEQ_P^2}{t}\qquad\qquad\qquad t = \frac{f_e T_e}{\Omega_e}
\end{equation*}

Having many detectors factors into the $t$, though they can factor into any of
$T_e$ or $\Omega_e$. In all cases, changing from 1 detector to $N$
detectors changes the integration time from $T_e \xrightarrow{N} NT_e$, since
each of the $N$ detectors is integrating. Additionally, $N$ detectors can map
out the sky $N$ times as quickly, so as long as the regions never overlap,
$\Omega_e \xrightarrow{N} N\Omega_e$. In this case, the integration time per
steradian scales like

\begin{align*}
    t & \xrightarrow{N} \frac{f_e N T_e}{N \Omega_e} = \frac{f_e T_e}{\Omega_e} \\
    t & \xrightarrow{N} t \qquad \text{for detectors tiled normal to the travel direction}
\end{align*}

and the experiment map sensitivity is not improved, but the experiment covers
a larger area in the same period of time. This scenario is relevant for
detectors that are tiled in a direction normal to the direction of travel
along the sky.

For the scenario where the $N - 1$ additional detectors are integrating a
region that the experiment has already covered, then the behavior is
different. This scenario is relevant for detectors that are tiled in a
direction parallel to the direction of travel along the sky. In this case, the
additional detectors are not measuring new solid angles, so the experimental
area does not scale with number of detectors, $\Omega_e \xrightarrow{N}
\Omega_e$. However, the integration time always scales up, $T_e
\xrightarrow{N} N T_e$. Thus, in this case, the integration time per steradian
scales like

\begin{align*}
    t & \xrightarrow{N} \frac{f_e N T_e}{\Omega_e} = N \frac{f_e T_e}{\Omega_e} \\
    t & \xrightarrow{N} N t \qquad \text{for detectors tiled parallel to the travel direction}
\end{align*}

Note that the scaling, and thus the integration time per solid angle, depends
on the scan strategy.

Next we consider the case where the beams of some adjacent detectors overlap
on the sky. The effects of this depends on where the noise originates from.
When the beams do not overlap, there is no need to distinguish the source of
noise, and noise from any origin may be treated similarly. We will first
consider noise that originates from the detector (e.g. phonon noise).
Following that we will consider noise that originates from the sky (e.g.
photon noise).

\textbf{Intrinsic Detector Noise}

For noise the originates in the detector, the noise is independent of the
signal. The noise in each pixel is independent from every other pixel. So in
this case, if there is overlap in the beams, then the same signal is measured
in more than one pixel. Since we are ignoring photon noise (natural variations
in the signal), the signal is always equal to its mean value. In this case,
the overlaps are essentially multiple measurements of the same signal, each
measurement with uncorrelated noise. Thus, this scenario is no different from
the overlapping region being visited twice at two different time periods, a
scenario that was discussed above. In this case, the integration time scales
with the number of detectors, regardless of the beam overlap.

\begin{equation*}
    t \xrightarrow{N} N t
\end{equation*}

\textbf{Sky Noise}

For noise that originates in the sky, the noise and signal are correlated. We
again model the signal as the mean, and now the noise is the variance. For
regions of the sky that are not overlapping, each instance of noise is a
realization of the 0-mean Gaussian with variance equal to the sky variance.
For regions that are overlapping, both detector pixels will sample the same
region and thus the noise realization will be identical. Thus we have only 1
sample of the noise in the overlap region and we cannot integrate down the
noise, even though we have multiple measurements, because the measurements are
correlated. For the regions without overlap, everything works as described in
the above general case. So we have

\begin{align*}
    t & \xrightarrow{N} t \quad \text{for overlapping regions} \\
    t & \xrightarrow{N} N t \quad \text{for non-overlapping regions}.
\end{align*}

A convenient way of approximating this is by replacing the actual number of
pixels with an effective number of pixels. We simply take the width of the
beam of the full array and divide by the beam width of a single pixel to get
the effective number of independent pixels along that direction
$N_\mathrm{eff}$. We note that this is only necessary for detectors tiled
parallel to the direction of the scan, since for perpendicularly tiled
detectors the integration time per solid angle does not scale with number of
detectors, so the transformation $N \to N_\mathrm{eff}$ makes no difference.

\section{Pixel Noise}
\label{sec:pixel_noise}

Let us use our map sensitivity to estimate the noise in each pixel of the sky,
which depends on our pixelization scheme. Equal-area pixelization schemes are
the most tractable, including HealPIX. Each pixel has a fixed angular size,
$\Omega_p$. The map sensitivity tells us how much noise is in a region of
some angular size, so we simply divide the map sensitivity by the pixel size
to get the pixel noise $N_p$,

\begin{equation}
    N_p = \frac{m}{\sqrt{\Omega_p}} = \frac{1}{\sqrt{\Omega_p t}} \left( \sqrt{\frac{2}{f_Q}} \frac{\unit{\sqrt{s}}}{\unit{Hz^{-1/2}}} \right) \frac{1}{\eta \tau} \frac{\NEP}{\left( \D{P}{T} \right)} \qquad\qquad \left[ \unit{\KCMB \sqrt{pixel}} \right]
\end{equation}

where we have included the unitless ``pixel'' to indicate the scaling of the
pixel noise with the number of pixels. Note that the units justify this, since
the units of $\Omega_p$ may be written $[\Omega_p] = \unit{sr/pixel}$. To get
the noise in a single pixel, simply divide by 1 pixel. To get the average
noise level in a region $N_\mathrm{pix}$ pixels large, simply divide by
$\sqrt{N_\mathrm{pix}}$.

\section{Map Sensitivity in Angular Units}
\label{sec:map_sensitivity_in_angular_units}

Frequently features are measured in units of radians (or degrees, arcminutes,
etc.) rather than steradians. Thus, we would like to know what the map
sensitivity for such features are. We suppose that a feature described by some
angular size $\theta$ is actually a spherical cap with angular diameter
$\theta$. Note that the opening angle for a spherical cap with angular
diameter $\theta$ is $\theta/2$, i.e. suppose the spherical cap is centered on
the $z$-axis, then the polar angle between the $z$-axis and the edge of the
spherical cap is $\theta/2$.

The solid angle of this spherical cap is

\begin{align}
    \Omega(\theta) & = \int \dd\Omega = \int_0^{\theta/2} \sin \theta'\ \dd\theta' \int \dd\phi \notag \\
    \Omega(\theta) & = 2\pi \left[ 1 - \cos{\left(\nicefrac{\theta}{2}\right)} \right]
\end{align}

We note that in the small angle limit,

\begin{equation*}
    \Omega(\theta) \simeq \frac{\pi \theta^2}{4}
\end{equation*}

as expected, since it reduces to a flat circle in this limit.

To get the NEQ in radians, we simply use this expression for $\Omega_e$ in Eq.~\ref{eq:m},

\begin{align}
    m & = \sqrt{\frac{2\pi \left[ 1 - \cos{\left(\nicefrac{\theta}{2}\right)} \right]}{f_sT_e}} \left( \sqrt{\frac{2}{f_Q}} \frac{\unit{\sqrt{s}}}{\unit{Hz^{-1/2}}} \right) \frac{1}{\eta \tau}\frac{\NEP}{\left( \D{P}{T} \right)} & \left[ \unit{\KCMB \cdot rad} \right]\\
    m & \simeq \frac{\sqrt{\pi} \theta}{2 \sqrt{f_sT_e}} \left( \sqrt{\frac{2}{f_Q}} \frac{\unit{\sqrt{s}}}{\unit{Hz^{-1/2}}} \right) \frac{1}{\eta \tau}\frac{\NEP}{\left( \D{P}{T} \right)} & \left[ \unit{\KCMB \cdot rad} \right]
\end{align}

\section{PIPER-like Experiment}
\label{sec:piper_like_experiment}

Let us estimate some of these parameters for a \piper-like
experiment. The parameters for the \piper\ experiment are listed in the table
below. All values are taken from the 2014 \piper\ proposal.

\begin{minipage}{\textwidth}
\begin{center}
\begin{tabular}{|c|cccc|}\hline
    Frequency & 200 GHz & 270 GHz & 350 GHz & 600 GHz\\
    Bandwidth $\delta\nu/\nu$ & 0.3 & 0.3 & 0.16 & 0.1 \\ \hline
    $\eta$ & 0.9 & 0.9 & 0.7 & 0.5 \\
    $\tau$ & 0.55 & 0.52 & 0.5 & 0.42 \\ \hline
    Single-pixel Area $\left[\unit{m^2}\right]$ & $1.3 \times 10^{-6}$\footnote{$\unit[1135]{\mu m} \times \unit[1135]{\mu m}$} & $1.3 \times 10^{-6}$ & $1.3 \times 10^{-6}$ & $1.3 \times 10^{-6}$ \\
    $f/\mathrm{N}$\footnote{Single pixel. At the detector.} & 1.6 & 1.6 & 1.6 & 1.6 \\
    $\Omega$\footnote{Single pixel. At the detector.} $\left[ \unit{sr} \right]$ & 0.2424 & 0.2424 & 0.2424 & 0.2424 \\
    $A\Omega$ $\left[ \unit{m^2\ sr} \right]$ & $3.1\times 10^{-7}$ & $3.1\times 10^{-7}$ & $3.1\times 10^{-7}$ & $3.1\times 10^{-7}$ \\ \hline
    NEP\footnote{Single pixel. At the detector.} (phonon) $\left[ \unit{\frac{W}{\sqrt{Hz}}} \right]$ & $4 \times 10^{-18}$ & $4 \times 10^{-18}$ & $4 \times 10^{-18}$ & $4 \times 10^{-18}$ \\
    NEP\footnote{Single pixel. At the detector.} (photon) $\left[ \unit{\frac{W}{\sqrt{Hz}}} \right]$ & $7 \times 10^{-18}$ & $8 \times 10^{-18}$ & $11 \times 10^{-18}$ & $13 \times 10^{-18}$ \\ \hline
    Independent Rows\footnote{Rows on the sky, not electronically addressed.} (of 32) & 14 & 18 & 21 & 27 \\
    Independent Columns\footnote{Columns on the sky, not electronically addressed.} (of 40) & 17 & 22 & 27 & 34 \\
    Independent Pixels (of 5120) & 943 & 1550 & 2270 & 3760 \\ \hline
    $f_Q$ & 4/17 & 4/17 & 4/17 & 4/17 \\
    $f_U$ & 4/17 & 4/17 & 4/17 & 4/17 \\
    $f_V$ & 13/17 & 13/17 & 13/17 & 13/17\\
    \hline
\end{tabular}
\end{center}
\end{minipage}

We must tally up the integration time for one flight of the experiment. The
detectors are oriented such that the array is aligned with the Earth's
horizon, with 40 detectors in the azimuthal direction and 32 detectors in the
polar direction. We will only account for measurements during the night.
\piper's scan strategy is to simply spin in azimuth at a constant rate and let
the sky rotation cover the sky. The spin rate is such that there is a 1/3
overlap from one circle on the sky to the next. With this scan strategy,
\piper\ will cover about 50\% of the sky in a single overnight flight. \piper\
has 4 co-pointed arrays of $32 \times 40$ pixels, for a total of 5120
detectors.

The base integration time per solid angle, not accounting for overlap factors,
is $t_0 = (\unit[10]{hrs})/(\unit[0.5 \cdot 4\pi]{sr}) = \unit[5700]{s/sr}$.
We now account for overlap factors.

We first note that since the 4 arrays are co-pointed, they always observe the
same region. Thus, all 4 arrays has 4 times the integration time of 1 array
but will cover the same area, resulting in a factor of 4 on the integration
time per solid angle. The 4 arrays gives us an overlap factor of
$f_\mathrm{arrays} = 4$.

Next we note that a single column (in the sense of the sky, not the electronic
addressing) of detectors will cover the same solid angle as the full array,
since the remaining 39 columns are simply following the first (see
Section~\ref{sec:multiple_detectors}). So the 40 rows gives us a overlap
factor $f_\mathrm{rows}^\mathrm{phonon} = 40$. However, this is valid only for
the phonon noise, since the beams for adjacent pixels on the sky overlap. For
photon noise, we have only between 14 and 27 independent columns, so the
overlap factor is only between 14 and 27 for photon noise.

The overlap of 1/3 of one circle to the next as the payload rotates means that
the sky coverage is such that we are effectively alternating stripes with 1/3
the array beam height with overlap factors alternating between 1 and 2. So we
have an effective overlap factor $f_\mathrm{scan} = 3/2$.

The total integration time per solid angle is then computed as

\begin{equation*}
    t = f_\mathrm{arrays} \cdot f_\mathrm{rows}^X \cdot f_\mathrm{scan} \cdot t_0 \qquad\qquad X \in \left\{ \mathrm{phonon}, \mathrm{photon} \right\}
\end{equation*}

which gives us

\begin{center}
\begin{tabular}{|c|cccc|}\hline
    Frequency & 200 GHz & 270 GHz & 350 GHz & 600 GHz \\ \hline
    $t_\mathrm{phonon}$ \quad $\left[ \unit[10^6]{s/sr} \right]$ & 1.1 & 1.1 & 1.1 & 1.1 \\
    $t_\mathrm{photon}$ \quad $\left[ \unit[10^6]{s/sr} \right]$ & 0.47 & 0.61 & 0.73 & 0.94 \\ \hline
\end{tabular}
\end{center}

We now have all the figures we need for map sensitivity. We repeat a few to
make a more convenient table.

\begin{minipage}{\textwidth}
\begin{center}
\begin{tabular}{|c|cccc|}\hline
    Frequency & 200 GHz & 270 GHz & 350 GHz & 600 GHz \\ \hline
    $\eta$ & 0.9 & 0.9 & 0.7 & 0.5 \\
    $\tau$ & 0.55 & 0.52 & 0.5 & 0.42 \\ \hline
    NEP\footnote{Single pixel. At the detector.} (phonon) $\left[ \unit{\frac{W}{\sqrt{Hz}}} \right]$ & $4 \times 10^{-18}$ & $4 \times 10^{-18}$ & $4 \times 10^{-18}$ & $4 \times 10^{-18}$ \\
    NEP\footnote{Single pixel. At the detector.} (photon) $\left[ \unit{\frac{W}{\sqrt{Hz}}} \right]$ & $7 \times 10^{-18}$ & $8 \times 10^{-18}$ & $11 \times 10^{-18}$ & $13 \times 10^{-18}$ \\ \hline
    $\D{P}{T}$ \quad $\left[ \frac{\unit{W}}{\unit{\KCMB}} \right]$ & 5.5 & 6.6 & 2.34 & 1.6 \\ \hline
    $f_Q$ & 4/17 & 4/17 & 4/17 & 4/17 \\
    $f_U$ & 4/17 & 4/17 & 4/17 & 4/17 \\
    $f_V$ & 13/17 & 13/17 & 13/17 & 13/17\\ \hline
    $\NEQ/\mathrm{U}$\footnote{Single detector. On the sky.} (phonon) \quad $\left[ \unit{\mu \KCMB \sqrt{s}} \right]$ & 299 & 259 & 840 & 15761 \\
    $\NEV$ \quad (phonon) $\left[ \unit{\mu \KCMB \sqrt{s}} \right]$ & 166 & 144 & 466 & 8743 \\
    $\NEQ/\mathrm{U}$ (photon) \quad $\left[ \unit{\mu \KCMB \sqrt{s}} \right]$ & 523 & 518 & 2311 & 51224 \\
    $\NEV$ \quad (photon) $\left[ \unit{\mu \KCMB \sqrt{s}} \right]$ & 290 & 287 & 1282 & 28414 \\ \hline
    $t$ (phonon) \quad $\left[ \unit[10^6]{s/sr} \right]$ & 1.1 & 1.1 & 1.1 & 1.1 \\
    $t$ (photon) \quad $\left[ \unit[10^6]{s/sr} \right]$ & 0.47 & 0.61 & 0.73 & 0.94 \\ \hline
    $m_{Q/U}$ (phonon) \quad $\left[ \unit{\mu \KCMB \sqrt{sr}} \right]$ & 0.28 & 0.25 & 0.80 & 15 \\
    $m_V$ (phonon) \quad $\left[ \unit{\mu \KCMB \sqrt{sr}} \right]$ & 0.16 & 0.14 & 0.44 & 8.3 \\
    $m_{Q/U}$ (phonon) \quad $\left[ \unit{\mu \KCMB \sqrt{sr}} \right]$ & 0.76 & 0.66 & 2.7 & 53 \\
    $m_V$ (photon) \quad $\left[ \unit{\mu \KCMB \sqrt{sr}} \right]$ & 0.42 & 0.37 & 1.5 & 29 \\ \hline
    $m_{Q/U}$ (total) \quad $\left[ \unit{\mu \KCMB \sqrt{sr}} \right]$ & 0.81 & 0.71 & 2.8 & 55 \\
    $m_V$ (total) \quad $\left[ \unit{\mu \KCMB \sqrt{sr}} \right]$ & 0.45 & 0.39 & 1.6 & 30 \\ \hline
\end{tabular}
\end{center}
\end{minipage}


\appendix
\newpage

\section{Relationship between Integration Time and Bandwidth}
\label{sec:bw}

Suppose we integrate a signal $x(t)$ for a period of time $T$. This is
equivalent to filtering the signal with a rect filter,

\begin{equation}
    h(t) = \rect(t/T) = u(t + \nicefrac{T}{2}) - u(t - \nicefrac{T}{2}),
\end{equation}

where $u(t)$ is the Heaviside step function. In harmonic space, this filter is

\begin{equation}
    H(f) = F\{ h(t) \} = T \sinc{( f T )} = T \frac{\sin(\pi f T)}{\pi f T}
\end{equation}

where we define our fourier transform as

\begin{equation*}
    F \{ x(t) \} = \int_{-\infty}^\infty \dd f\ x(t) \exp(-2\pi i f t)
\end{equation*}

The bandwidth of a signal is conventionally defined as the distance in
frequency space between the first positive and first negative node. We note
that the sinc function is symmetric and has its first node at $f = 1/T$,
so the bandwidth is $B = 2/T$. Thus the relationship between integration time
and bandwidth is

\begin{align*}
    \unit[T]{seconds\ integration\ time} & \longleftrightarrow \unit[\frac{2}{T}]{Hz\ bandwidth} \\
    \unit[1]{second\ integration\ time} & \longleftrightarrow \unit[2]{Hz\ bandwidth}
\end{align*}

and so the conversion between seconds of integration time and Hz of bandwidth is

\begin{equation}
    1 = \frac{\unit[1]{s}}{\unit[\frac{1}{2}]{Hz^{-1}}} = \frac{\unit[2]{s}}{\unit[1]{Hz^{-1}}}
\end{equation}

As a final note, the relationship between bandwidth and integration time is
inversely proportional. As the integration time increases, we integrate a
smaller bandwidth. This is intuitively correct, since as the bandwidth
decreases, the noise will decrease, and as the integration time increases, the
noise will decrease.

\bibliography{noise}
\bibliographystyle{plain}

\end{document}
