%%% ilc.tex ---

\documentclass[twoside,10pt]{article}

% \usepackage{epsfig}
\usepackage{color}
\usepackage{fancyhdr}
\usepackage{rotating}
\usepackage{multirow}
\usepackage{graphicx}
\usepackage{amssymb}
\usepackage{nicefrac}
\usepackage{cool}
\usepackage{mathrsfs}
\usepackage{units}
\usepackage{amsmath}
\usepackage{subfig}
\usepackage{bm}
\Style{DSymb={\mathrm d},IntegrateDifferentialDSymb=\text{d}}

% define formatting
% A4 paper is 210 x 297 mm
% pt  point         (1 in = 72.27 pt)
% pc  pica          (1 pc = 12 pt)
% in  inch          (1 in = 25.4 mm)
% bp  big point     (1 in = 72 bp)
% cm  centimetre    (1 cm = 10 mm)
% mm  millimetre
% dd  didot point   (1157 dd = 1238 pt)
% cc  cicero        (1 cc = 12 dd)
% sp  scaled point  (65536 sp = 1 pt)
% em  the width of the letter M in the current font
% ex  the height of the letter x in the current font

\setlength{\textheight}{11.in}
\setlength{\textwidth}{8.5in}

\setlength{\topmargin}{0.in}
\setlength{\headheight}{\baselineskip}
\setlength{\headsep}{2\baselineskip}

\addtolength{\textheight}{-\headheight}
\addtolength{\textheight}{-\headsep}
\addtolength{\textheight}{-2.in}
\addtolength{\textwidth}{-2.in}


\setlength{\oddsidemargin}{0.in}
\setlength{\evensidemargin}{0.in}

\setlength{\parindent}{0.em}
\setlength{\parskip}{0.75\baselineskip}

\def\boxwidth{\textwidth}
\def\doctitle{Internal Linear Combination Method for Foreground Subtraction}
\def\myname{Justin Lazear}      % Primary author defined here
\def\myemail{\texttt{jlazear@pha.jhu.edu}} % Primary author email
\def\piper{\textsc{Piper}}
\def\tablefont{\footnotesize}

% \renewcommand{\thefootnote}{\fnsymbol{footnote}}

%% \pagestyle{fancy}
%% \fancyhead{}
%% \fancyhead[LE,RO]{{\it {\large \thepage}}}
%% \fancyhead[RE]{{\it {\large \myname{}}}}
%% \fancyhead[LO]{{\it {\large \doctitle{}}}}
%% \fancyfoot{}

%%%%%%%%%%%%%%%%%%%%%%%%%%%%%%%%%%%%%%%%%%%%%%%%%%%%%%%%%%%%%%%%%%%%%%

% Define some math macros for convenience

% Note that most necessarily math macros are defined by the cool
% package, so this section is sparse.

\DeclareMathOperator{\num}{num}
\DeclareMathOperator{\vvar}{var}
\DeclareMathOperator{\cvar}{cov}
\DeclareMathOperator{\bcvar}{\mathbf{cov}}
\DeclareMathOperator{\diagop}{diag}
\DeclareMathOperator*{\argmin}{arg\,min}

\newcommand{\ve}[1]{\mathbf{#1}}
\newcommand{\Avg}[1]{\left< #1 \right>}
\newcommand{\pp}[1]{\left( #1 \right)}
\newcommand{\pb}[1]{\left[ #1 \right]}
\newcommand{\pc}[1]{\left\{ #1 \right\} }
\newcommand{\dd}[0]{\mathrm{d}}
\newcommand{\Tip}[0]{T_i(p)}
\newcommand{\Tcmbp}[0]{T_\mathrm{CMB}(p)}
\newcommand{\Rip}[0]{R_i(p)}
\newcommand{\wip}[0]{w_i(p)}
\newcommand{\Rjp}[0]{R_j(p)}
\newcommand{\zjp}[0]{\zeta_j(p)}
\newcommand{\hTp}[0]{\hat{T}(p)}
\newcommand{\Ti}[0]{T_i}
\newcommand{\Tcmb}[0]{T_\mathrm{CMB}}
\newcommand{\Ri}[0]{R_i}
\newcommand{\Rj}[0]{R_j}
\newcommand{\zj}[0]{\zeta_j}
\newcommand{\hT}[0]{\hat{T}}
\newcommand{\var}[1]{\vvar{\left(#1\right)}}
\newcommand{\cov}[1]{\cvar{\left(#1\right)}}
\newcommand{\bcov}[1]{\bcvar{\left(#1\right)}}
\newcommand{\covT}[1]{\cvar^T{\left(#1\right)}}
\newcommand{\bcovT}[1]{\bcvar^T{\left(#1\right)}}
\newcommand{\bone}[0]{\mathbf{1}}
\newcommand{\diag}[1]{\diagop{\left( #1 \right)}}
\newcommand{\bz}[0]{\bm{\zeta}}
\newcommand{\pderivht}[1]{\frac{\partial \vvar{\hat{T}}}{\partial #1}}
\newcommand{\varhT}[0]{\vvar{\hat{T}}}
\newcommand{\vw}[0]{\ve{w}}
\newcommand{\vT}[0]{\ve{T}}
\newcommand{\bC}[0]{\mathbf{C}}
\newcommand{\bF}[0]{\mathbf{F}}
\newcommand{\bG}[0]{\mathbf{G}}
\newcommand{\vX}[0]{\ve{X}}


%%%%%%%%%%%%%%%%%%%%%%%%%%%%%%%%%%%%%%%%%%%%%%%%%%%%%%%%%%%%%%%%%%%%%%

\title{\doctitle}
\author{\myname}
\date{Jan 7, 2014}

\begin{document}
\maketitle

\section{Internal Linear Combination method of foreground subtraction}
\label{sec:ilc}

Consider a measurement of the full CMB sky in $k$ different frequency bands.
The product of such a measurement is $k$ temperature maps

\begin{equation}
    \Tip = \text{Temperature map of frequency band }\nu_i\text{ with pixel index }p
\end{equation}

where $i$ is the frequency band index with $i = 1, \dots, k$ and $p$ is the
pixel index with $p = 1, \dots, N$. We will choose to represent the maps in
thermodynamic temperature units\footnote{Thermodynamic temperature units are
defined relative to the Planck distribution, and are a measure of power per
unit area. The key point to keep in mind that the thermodynamic temperature
units are referencing a blackbody (Planck) spectrum, so any additional
information required to get to the desired units must be pulled from the
properties of the map (e.g. area).

Sky maps are usually given in terms of spectral intensity, i.e. energy per
unit area per unit solid angle per unit frequency, with SI unit
$\unit{\frac{W}{m^2\,sr\,Hz}}$. Commonly, intensity is given in terms of
mega-Janskies per steradian, where $\unit[1]{Jy} = \unit[10^{-26}]{\frac{W}{m^2\,Hz}}$, so
$\unit[1]{\frac{w}{m^2\,sr\,Hz}} = \unit[10^{20}]{\frac{MJy}{sr}}$.

Spectral intensity $I_\nu$ is related to thermodynamic temperature units $T$
by the Planck distribution,
\begin{equation}
    I_\nu = \frac{2 h \nu^3}{c^2} \frac{1}{\exp{\left(h\nu/k_B T \right)} - 1} = B_\nu(T).
\end{equation}
From inspection, we can see that $T$ has units of Kelvin, but encodes the
same information as spectral intensity.

The last common unit is antenna temperature $T_A$, which is the low-frequency
limit of the above relation,
\begin{equation}
    I_\nu = \frac{2 k_B \nu^2}{c^2} T_A,
\end{equation}
where we note that all we have done is expand the inverse exponential term in
the $h\nu/k_B T \ll 1$ limit. The antenna temperature also has units of
Kelvin.

We note that the conversions between spectral intensity and the
temperatures are frequency-dependent, and do not simply have a numerical
value. Additionally, since the CMB spectrum is a blackbody (ignoring spectral
distortions), its thermodynamic temperature is constant for all frequencies.}.

Such maps are usually encoded in the HEALpix\cite{gorski_healpix_2005} format.
Care must be taken to ensure that all maps that are used have the same number
of pixels (typically $N_\mathrm{side} = 512$ for WMAP) and that all maps have
been smoothed to the same resolution (typically $1^\circ$).

The temperature maps can (theoretically) be decomposed into a CMB component
and a residual component
\begin{equation}
    \Tip = \Tcmbp + \Rip,
\end{equation}
where the $\Tcmbp$ component does not dependent on frequency (since we have
chosen thermodynamic temperature units), and the $\Rip$ component encodes
\emph{all} sources of signal that are not from the CMB. Since the CMB
component will be constant in frequency, we expect that we can combine the
various maps in different frequencies to construct an estimator of the CMB map
$\Tcmbp$. We construct an estimator using a linear combination of the maps in
different frequency bands with to-be-determined weights
$\wip$\cite{bennett_first-year_2003, eriksen_foreground_2004,
hinshaw_three-year_2007},
\begin{align}
    \hTp & = \sum_{i=1}^k \wip \Tip \\
    & = \sum_i \wip \left[ \Tcmbp + \Rip \right] \notag \\
    \hTp & = \Tcmbp \sum_i \wip + \sum_i \wip \Rip.
\end{align}
We note that there are $\num{(\wip)} = Nk$ unknown parameters in our estimator.
In order for the estimator $\hTp$ to have unity gain in $\Tcmbp$, we must have
\begin{equation}
    \sum_i \wip = 1 \qquad \forall\ p
\end{equation}
which provides $N$ constraint equations. This results in
\begin{equation}
    \hTp = \Tcmbp + \sum_{i=1}^k \wip \Rip.
\end{equation}

Further properties of the estimator $\hTp$ will be determined by how we choose
to constrain the remaining $(N-1)k$ degrees of freedom.

Let us now introduce a matrix notation for this system.

Define vectors as column vectors of the frequency bands, so

\begin{equation*}
    \vT^T = \left( T_1(p), \dots, T_k(p) \right) \quad \text{and} \quad
    \vw^T = \left( w_1(p), \dots, w_k(p) \right),
\end{equation*}

then we may write our estimator as

\begin{equation}
    \hT = \vw^T \vT
\end{equation}

and the unity gain constraint as

\begin{equation}
    \ve{1}^T \vw = 1
\end{equation}

\section{Constant weighting factors}
\label{sec:constant_weighting_factors}

Suppose the weight factors $\vw(p)$ were uniform across the entire map, so
$\vw(p) = \vw$. We note in this case that our estimator $\hT$ is a
linear function with regards to $\vT$, so then the variance of $\hT$ has a
particularly simple form

\begin{equation}
    \varhT = \vw^T \cov{\vT, \vT} \vw
\end{equation}

where $\left(\cov{\vT, \vT}\right)_{ij} = \cov{T_i, T_j}$\footnote{
Expectation values are defined over the pixels, so
\begin{equation*}
    \Avg{T_i} = \sum_{p=1}^N T_i(p)
\end{equation*}
then
\begin{equation*}
    \cov{T_i, T_j} = \Avg{T_i T_j} - \Avg{T_i}\Avg{T_j}.
\end{equation*}
It is worth noting that $\cov{\vT, \vT} = \cov{\vT, \vT}^T$ and
$\cov{\vT, \vT} \succeq 0$, i.e. the covariance matrix is symmetric and
positive semi-definite.} is the covariance matrix of $\vT$.

We choose to optimize $\vw$ so that $\varhT$ is minimized, i.e.

\begin{equation}
    \vw^* = \argmin_{\vw} \left( \varhT(\vw; \vT) \right)
\end{equation}

We will do this in two spaces: the raw temperature map space $\pc{\vT}$, and
the foreground-background space $\pc{\Tcmb, \ve{R}}$. The first solution will
be used to actually perform the foreground cleaning, while the second will
give us a sense of what the algorithm is doing. Let us proceed with the first
solution.

\subsection{Raw temperature map space}
\label{sub:raw_temperature_map_space}

Let $\bC = \cov{\vT, \vT}$, i.e. $\bC_{ij} = \cov{T_i, T_j}$.
Then we may write our minimization problem as

\begin{equation}
    \varhT = \vw^T \bC \vw \quad \text{subject to} \quad \bone^T \vw = 1
\end{equation}

which may be solved with Lagrange multipliers. Define the function

\begin{equation}
    L = \varhT + \lambda \left( \bone^T \vw - 1 \right).
\end{equation}

Then the solution to our optimization problem is given by the set of equations

\begin{gather}
    \frac{\partial L}{\partial \vw^T} = 2\bC \vw + \lambda \bone = 0 \label{eq:dLdw}\\
    \bone^T \vw = 1 \label{eq:constraint}
\end{gather}

Solving Eq.~\eqref{eq:dLdw} for $\vw$

\begin{equation*}
    \vw = -\frac{\lambda}{2} \bC^{-1} \bone
\end{equation*}

and plugging it into Eq.~\eqref{eq:constraint} allows us to solve for $\lambda$,

\begin{align}
    \bone^T \vw & = -\frac{\lambda}{2} \bone^T \bC^{-1} \bone = 1 \notag \\
    \lambda & = -2\left( \bone^T \bC^{-1} \bone \right)^{-1}
\end{align}

and then we may substitute into $\vw$,

\begin{equation}
    \vw^* = \frac{\bC^{-1} \bone}{\left( \bone^T \bC^{-1} \bone \right)}
\end{equation}

We note that

\begin{equation}
    w_i^* = \frac{\sum_j \mathrm{C}^{-1}_{ij}}{\sum_{jk}\mathrm{C}^{-1}_{jk}}
\end{equation}

which matches Eriksen et al\cite{eriksen_foreground_2004}.

$\bC$ is a $k \times k$ matrix, where $k$ is the number of frequency bands,
which is typically order unity. Thus, computing $\vw^*$ from $\bC$ requires
the inversion of a small matrix and the sum of its components and may be done
cheaply. Far more expensive is the computation of each element of the matrix
$\mathrm{C}_{ij} = \sum_{p=1}^N T_i(p) T_j(p) - \sum_{p=1}^N T_i(p) \sum_{p=1}^N T_j(p)$,
which is $\mathcal{O}(4N)$. Given that matrix inversion of $\bC$ is
pessimistically $\mathcal{O}(k^3)$ and $N = 12N_\mathrm{side}^2$, our total
complexity is $\mathcal{O}(48N_\mathrm{side}^2 + k^3)$.

\subsection{Foreground-background space}
\label{sub:foreground_background_space}

Let us now perform the same optimization supposing we already know the
decomposition between residual (foreground) and CMB (background). The
optimization procedure will produce the same set of equations to solve,
Eqs.~\eqref{eq:dLdw} and \eqref{eq:constraint}.

\begin{gather*}
    \frac{\partial L}{\partial \vw^T} = 2\bC \vw + \lambda \bone = 0 \label{eq:dLdw}\\
    \bone^T \vw = 1 \label{eq:constraint}
\end{gather*}

We note that the covariance matrix may be decomposed into foreground and
background components, since $\vT = \Tcmb \bone + \ve{R}$,

\begin{align}
    \bC & = \cov{\Tcmb \bone + \ve{R}, \Tcmb \bone + \ve{R}} \notag \\
    & = \var{\Tcmb} \bone \bone^T + \bcov{\Tcmb, \ve{R}} \bone^T + \bone \bcov{\Tcmb, \ve{R}}^T + \cov{\ve{R}, \ve{R}} \notag \\
    \bC & = \sigma_B^2 \bone \bone^T + \vX \bone^T + \bone \vX^T + \bF
\end{align}

where we've defined $\sigma_B^2 \equiv \var{\Tcmb}$,
$\vX \equiv \bcov{\Tcmb, \ve{R}}$\footnote{Explicitly,
$X_i = \cov{\Tcmb, R_i}$.}, and $\bF = \cov{\ve{R}, \ve{R}}$. So our system of
equations becomes

\begin{gather}
    2\sigma_B^2 \bone + 2\vX + 2 \bone \vX^T \vw + 2\bF \vw + \lambda \bone = 0 \\
    \bone^T \vw = 1 \notag
\end{gather}

which may be solved in an identical manner,

\begin{align}
    \vw & = -\left( \frac{\lambda}{2} + \sigma_B^2 \right) (\bF + \bone \vX^T)^{-1} \bone - (\bF + \bone \vX^T)^{-1} \vX \notag \\
    \vw & = -\left( \frac{\lambda}{2} + \sigma_B^2 \right) \bG^{-1} \bone - \bG^{-1} \vX
\end{align}

where we've defined $\bG \equiv \bF + \bone \vX^T$. So

\begin{equation}
    \frac{\lambda}{2} + \sigma_B^2 = -\frac{1 + (\bone^T \bG^{-1} \vX)}{(\bone^T \bG^{-1} \bone)}
\end{equation}

and our solution is

\begin{equation}
    \vw^* = \frac{1 + (\bone^T \bG^{-1} \vX)}{(\bone^T \bG^{-1} \bone)} \bG^{-1} \bone - \bG^{-1} \vX
\end{equation}

In component form, this is

\begin{equation}
    w_i^* = \frac{1 + \sum_{jk} \left( \bF + \bone \vX^T \right)^{-1}_{jk} X_j}{\sum_{jk} \left( \bF + \bone \vX^T \right)^{-1}_{jk}} \sum_{j} \left( \bF + \bone \vX^T \right)^{-1}_{ij} - \sum_{j} \left( \bF + \bone \vX^T \right)^{-1}_{ij} X_j
\end{equation}

which differs slightly from the result of Efstathiou et
al\cite{efstathiou_impact_2009} in that $\bF \to \bF + \bone \vX^T$.

\bibliography{ilc}
\bibliographystyle{plain}

\end{document}