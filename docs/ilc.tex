%%% ilc.tex ---

\documentclass[twoside,10pt]{article}

% \usepackage{epsfig}
\usepackage{color}
\usepackage{fancyhdr}
\usepackage{rotating}
\usepackage{multirow}
\usepackage{graphicx}
\usepackage{amssymb}
\usepackage{nicefrac}
\usepackage{cool}
\usepackage{mathrsfs}
\usepackage{units}
\usepackage{amsmath}
\usepackage{subfig}
\usepackage{bm}
\Style{DSymb={\mathrm d},IntegrateDifferentialDSymb=\text{d}}

% define formatting
% A4 paper is 210 x 297 mm
% pt  point         (1 in = 72.27 pt)
% pc  pica          (1 pc = 12 pt)
% in  inch          (1 in = 25.4 mm)
% bp  big point     (1 in = 72 bp)
% cm  centimetre    (1 cm = 10 mm)
% mm  millimetre
% dd  didot point   (1157 dd = 1238 pt)
% cc  cicero        (1 cc = 12 dd)
% sp  scaled point  (65536 sp = 1 pt)
% em  the width of the letter M in the current font
% ex  the height of the letter x in the current font

\setlength{\textheight}{11.in}
\setlength{\textwidth}{8.5in}

\setlength{\topmargin}{0.in}
\setlength{\headheight}{\baselineskip}
\setlength{\headsep}{2\baselineskip}

\addtolength{\textheight}{-\headheight}
\addtolength{\textheight}{-\headsep}
\addtolength{\textheight}{-2.in}
\addtolength{\textwidth}{-2.in}


\setlength{\oddsidemargin}{0.in}
\setlength{\evensidemargin}{0.in}

\setlength{\parindent}{0.em}
\setlength{\parskip}{0.75\baselineskip}

\def\boxwidth{\textwidth}
\def\doctitle{Internal Linear Combination Method for Foreground Subtraction}
\def\myname{Justin Lazear}      % Primary author defined here
\def\myemail{\texttt{jlazear@pha.jhu.edu}} % Primary author email
\def\piper{\textsc{Piper}}
\def\tablefont{\footnotesize}

% \renewcommand{\thefootnote}{\fnsymbol{footnote}}

%% \pagestyle{fancy}
%% \fancyhead{}
%% \fancyhead[LE,RO]{{\it {\large \thepage}}}
%% \fancyhead[RE]{{\it {\large \myname{}}}}
%% \fancyhead[LO]{{\it {\large \doctitle{}}}}
%% \fancyfoot{}

%%%%%%%%%%%%%%%%%%%%%%%%%%%%%%%%%%%%%%%%%%%%%%%%%%%%%%%%%%%%%%%%%%%%%%

% Define some math macros for convenience

% Note that most necessarily math macros are defined by the cool
% package, so this section is sparse.

\DeclareMathOperator{\num}{num}
\DeclareMathOperator{\vvar}{var}
\DeclareMathOperator{\cvar}{cov}
\DeclareMathOperator{\bcvar}{\mathbf{cov}}
\DeclareMathOperator{\diagop}{diag}

\newcommand{\ve}[1]{\mathbf{#1}}
\newcommand{\Avg}[1]{\left< #1 \right>}
\newcommand{\pp}[1]{\left( #1 \right)}
\newcommand{\pb}[1]{\left[ #1 \right]}
\newcommand{\pc}[1]{\left{ #1 \right} }
\newcommand{\dd}[0]{\mathrm{d}}
\newcommand{\Tip}[0]{T_i(p)}
\newcommand{\Tcmbp}[0]{T_\mathrm{CMB}(p)}
\newcommand{\Rip}[0]{R_i(p)}
\newcommand{\zip}[0]{\zeta_i(p)}
\newcommand{\Rjp}[0]{R_j(p)}
\newcommand{\zjp}[0]{\zeta_j(p)}
\newcommand{\hTp}[0]{\hat{T}(p)}
\newcommand{\Ti}[0]{T_i}
\newcommand{\Tcmb}[0]{T_\mathrm{CMB}}
\newcommand{\Ri}[0]{R_i}
\newcommand{\zi}[0]{\zeta_i}
\newcommand{\Rj}[0]{R_j}
\newcommand{\zj}[0]{\zeta_j}
\newcommand{\hT}[0]{\hat{T}}
\newcommand{\var}[1]{\vvar{\left(#1\right)}}
\newcommand{\cov}[1]{\cvar{\left(#1\right)}}
\newcommand{\bcov}[1]{\bcvar{\left(#1\right)}}
\newcommand{\covT}[1]{\cvar^T{\left(#1\right)}}
\newcommand{\bcovT}[1]{\bcvar^T{\left(#1\right)}}
\newcommand{\bone}[0]{\mathbf{1}}
\newcommand{\diag}[1]{\diagop{\left( #1 \right)}}
\newcommand{\bz}[0]{\bm{\zeta}}
\newcommand{\pderivht}[1]{\frac{\partial \vvar{\hat{T}}}{\partial #1}}
\newcommand{\varhT}[0]{\vvar{\hat{T}}}


%%%%%%%%%%%%%%%%%%%%%%%%%%%%%%%%%%%%%%%%%%%%%%%%%%%%%%%%%%%%%%%%%%%%%%

\title{\doctitle}
\author{\myname}
\date{Jan 7, 2014}

\begin{document}
\maketitle

\section{Internal Linear Combination method of foreground subtraction}
\label{sec:ilc}

Consider a measurement of the full CMB sky in $k$ different frequency bands.
The product of such a measurement is $k$ temperature maps

\begin{equation}
    \Tip = \text{Temperature map of frequency band }\nu_i\text{ with pixel index }p
\end{equation}

where $i$ is the frequency band index with $i = 1, \dots, k$ and $p$ is the
pixel index with $p = 1, \dots, N$. We will choose to represent the maps in
thermodynamic temperature units\footnote{Thermodynamic temperature units are
defined relative to the Planck distribution, and are a measure of power per
unit area. The key point to keep in mind that the thermodynamic temperature
units are referencing a blackbody (Planck) spectrum, so any additional
information required to get to the desired units must be pulled from the
properties of the map (e.g. area).

Sky maps are usually given in terms of spectral intensity, i.e. energy per
unit area per unit solid angle per unit frequency, with SI unit
$\unit{\frac{W}{m^2\,sr\,Hz}}$. Commonly, intensity is given in terms of
mega-Janskies per steradian, where $\unit[1]{Jy} = \unit[10^{-26}]{\frac{W}{m^2\,Hz}}$, so
$\unit[1]{\frac{w}{m^2\,sr\,Hz}} = \unit[10^{20}]{\frac{MJy}{sr}}$.

Spectral intensity $I_\nu$ is related to thermodynamic temperature units $T$
by the Planck distribution,
\begin{equation}
    I_\nu = \frac{2 h \nu^3}{c^2} \frac{1}{\exp{\left(h\nu/k_B T \right)} - 1} = B_\nu(T).
\end{equation}
From inspection, we can see that $T$ has units of Kelvin, but encodes the
same information as spectral intensity.

The last common unit is antenna temperature $T_A$, which is the low-frequency
limit of the above relation,
\begin{equation}
    I_\nu = \frac{2 k_B \nu^2}{c^2} T_A,
\end{equation}
where we note that all we have done is expand the inverse exponential term in
the $h\nu/k_B T \ll 1$ limit. The antenna temperature also has units of
Kelvin.

We note that the conversions between spectral intensity and the
temperatures are frequency-dependent, and do not simply have a numerical
value. Additionally, since the CMB spectrum is a blackbody (ignoring spectral
distortions), its thermodynamic temperature is constant for all frequencies.}.

Such maps are usually encoded in the HEALpix\cite{gorski_healpix_2005} format.
Care must be taken to ensure that all maps that are used have the same number
of pixels (typically $N_\mathrm{side} = 512$ for WMAP) and that all maps have
been smoothed to the same resolution (typically $1^\circ$).

The temperature maps can (theoretically) be decomposed into a CMB component
and a residual component
\begin{equation}
    \Tip = \Tcmbp + \Rip,
\end{equation}
where the $\Tcmbp$ component does not dependent on frequency (since we have
chosen thermodynamic temperature units), and the $\Rip$ component encodes
\emph{all} sources of signal that are not from the CMB. Since the CMB
component will be constant in frequency, we expect that we can combine the
various maps in different frequencies to construct an estimator of the CMB map
$\Tcmbp$. We construct an estimator using a linear combination of the maps in
different frequency bands with to-be-determined weights
$\zip$\cite{bennett_first-year_2003, eriksen_foreground_2004,
hinshaw_three-year_2007},
\begin{align}
    \hTp & = \sum_{i=1}^k \zip \Tip \\
    & = \sum_i \zip \left[ \Tcmbp + \Rip \right] \notag \\
    \hTp & = \Tcmbp \sum_i \zip + \sum_i \zip \Rip.
\end{align}
We note that there are $\num{(\zip)} = Nk$ unknown parameters in our estimator.
In order for the estimator $\hTp$ to have unity gain in $\Tcmbp$, we must have
\begin{equation}
    \sum_i \zip = 1 \qquad \forall\ p
\end{equation}
which provides $N$ constraint equations. This results in
\begin{equation}
    \hTp = \Tcmbp + \sum_{i=1}^k \zip \Rip.
\end{equation}

Further properties of the estimator $\hTp$ will be determined by how we choose
to constrain the remaining $(N-1)k$ degrees of freedom.

We note some statistical properties of the estimator.

\begin{gather*}
    \Avg{\hT} = \Avg{\Tcmb} + \sum_i \Avg{\zi \Ri} \\
    \varhT = \var{\Tcmb} + \sum_i \var{\zi \Ri} + \sum_i \cov{\Tcmb, \zi\Ri} + \sum_{i\neq j} \cov{\zi \Ri, \zj \Rj}
\end{gather*}

where the expectation value, variance, and covariance operators are defined
over the pixels\footnote{I.e.,

\begin{gather*}
    \Avg{x} = \frac{1}{N} \sum_{p=1}^N x(p) \\
    \var{x} = \cov{x, x} = \Avg{x^2} - \Avg{x}^2 \\
    \cov{x, y} = \Avg{xy} - \Avg{x}\Avg{y}
\end{gather*}
}.

We can write the variance of the estimator in a matrix formalism by expanding
the definition of the covariance operator to allow for vector\footnote{More
precisely, the covariance operator is a functional and we are extending it to
allow for a vector of discrete input functions. In this formalism, we think of
each map as a discrete function with a single argument $p$ and domain
$p = 1, \dots, N$.} arguments,

\begin{gather*}
    \bcov{a, \ve{x}} =
        \begin{pmatrix}
            \cov{a, x_1} \\
            \vdots \\
            \cov{a, x_n}
        \end{pmatrix}, \qquad \bcov{a, \ve{x}^T} =
        \begin{pmatrix}
            \cov{a, x_1} & \cdots & \cov{a, x_n}
        \end{pmatrix} = \bcov{a, \ve{x}}^T \\
    \cov{\ve{x}, \ve{y}^T} = \cov{\ve{x}^T, \ve{y}} = \cov{\ve{x}, \ve{y}^T}^T =
        \begin{pmatrix}
            \cov{x_1, y_1} & \cdots & \cov{x_1, y_n} \\
            \vdots         & \ddots & \vdots         \\
            \cov{x_n, y_1} & \cdots & \cov{x_n, y_n}
        \end{pmatrix}.
\end{gather*}

Then we define $\ve{v}^T = \left(\begin{smallmatrix}\Tcmb & \bm{\eta}^T\end{smallmatrix}\right) = \left( \Tcmb,\ \zeta_1R_1,\ \dots,\ \zeta_kR_k \right)$,
so

\begin{equation}
    \varhT = \bone^T \cov{\ve{v}, \ve{v}^T} \bone
\end{equation}

and we note that $\cov{\ve{v}, \ve{v}^T}$ has the form

\begin{equation}
    \cov{\ve{v}, \ve{v}^T} =
        \begin{pmatrix}
            \cov{\Tcmb, \Tcmb}      & \bcov{\Tcmb, \bm{\eta}^T}     \\
            \bcov{\Tcmb, \bm{\eta}} & \cov{\bm{\eta}, \bm{\eta}^T}
        \end{pmatrix}
\end{equation}

As a final note, all we can say about the estimator at this point is that it
is unity gain in $\Tcmb$ and that $\varhT \geq 0$ (since covariance
matrices are positive semi-definite). $\hT$ may be biased and its variance may
not be an accurate estimate of $\var{\Tcmb}$.

\section{Constant weighting factors}
\label{sec:constant_weighting_factors}

Suppose that the weight factors $\zip$ were uniform across the entire map, so
$\zip = \zi$. Then

\begin{equation}
    \cov{\Tcmb, \zi \Ri} = \zi \cov{\Tcmb, \Ri} \quad\text{and}\quad
    \cov{\zi\Ri, \zj\Rj} = \zi \zj \cov{\Ri, \Rj}
\end{equation}

so

\begin{equation}
    \bcov{\Tcmb, \bm{\eta}} = \diag{\bz} \bcov{\Tcmb, \ve{\Ri}} \quad\text{and}\quad
    \cov{\bm{\eta}, \bm{\eta}^T} = \diag{\bz} \cov{\ve{R}, \ve{R}^T} \diag{\bz}
\end{equation}

and $\varhT$ may be written

\begin{align}
    \varhT & =
        \begin{pmatrix}
            1 & \bz^T
        \end{pmatrix}
        \begin{pmatrix}
            \cov{\Tcmb, \Tcmb}     & \bcov{\Tcmb, \ve{R}^T} \\
            \bcov{\Tcmb, \ve{R}}   & \cov{\ve{R}, \ve{R}^T}
        \end{pmatrix}
        \begin{pmatrix}
            1 \\
            \bz
        \end{pmatrix} \notag \\
    \varhT & = \var{\Tcmb} + 2\bcov{\Tcmb, \ve{R}^T} \bz + \bz^T \cov{\ve{R}, \ve{R}^T} \bz \label{eq:varhT}
\end{align}

which we note is a convex\footnote{The positive semi-definiteness of
$\cov{\ve{R}, \ve{R}^T}$ implies that the function is convex.} quadratic form
in $\bz$, which therefore must have a unique minimum.

We may minimize $\varhT$ by finding the extremum of the RHS of the Eq.~\ref{eq:varhT}.

\begin{align}
    0 = \pderivht{\bz} & = \left( \pderivht{\zeta_1},\ \dots,\ \pderivht{\zeta_n} \right) = 2\bcov{\Tcmb, \ve{R}} + 2\bz^T \cov{\ve{R}, \ve{R}^T} \notag\\
    \bz^* & = -\cov{\ve{R}, \ve{R}^T}^{-1} \bcov{\Tcmb, \ve{R}}
\end{align}

which matches the result of Hinshaw 2007\cite{hinshaw_three-year_2007} (Eq.~10).

Does $\bz^*$ satisfy $\sum_i \zeta^*_i = \ve{1}^T \bz^* = -\ve{1}^T \cov{\ve{R}, \ve{R}^T}^{-1} \bcov{\Tcmb, \ve{R}} = 1$?

\bibliography{ilc}
\bibliographystyle{plain}

\end{document}