%%% cmb.tex ---

\documentclass[twoside,10pt]{article}

% \usepackage{epsfig}
\usepackage{color}
\usepackage{fancyhdr}
\usepackage{rotating}
\usepackage{multirow}
\usepackage{graphicx}
\usepackage{amssymb}
\usepackage{nicefrac}
\usepackage{cool}
\usepackage{mathrsfs}
\usepackage{units}
\usepackage{amsmath}
\usepackage{subfig}
\usepackage{hyperref}
\Style{DSymb={\mathrm d},IntegrateDifferentialDSymb=\text{d}}

% define formatting
% A4 paper is 210 x 297 mm
% pt  point         (1 in = 72.27 pt)
% pc  pica          (1 pc = 12 pt)
% in  inch          (1 in = 25.4 mm)
% bp  big point     (1 in = 72 bp)
% cm  centimetre    (1 cm = 10 mm)
% mm  millimetre
% dd  didot point   (1157 dd = 1238 pt)
% cc  cicero        (1 cc = 12 dd)
% sp  scaled point  (65536 sp = 1 pt)
% em  the width of the letter M in the current font
% ex  the height of the letter x in the current font

\setlength{\textheight}{11.in}
\setlength{\textwidth}{8.5in}

\setlength{\topmargin}{0.in}
\setlength{\headheight}{\baselineskip}
\setlength{\headsep}{2\baselineskip}

\addtolength{\textheight}{-\headheight}
\addtolength{\textheight}{-\headsep}
\addtolength{\textheight}{-2.in}
\addtolength{\textwidth}{-2.in}


\setlength{\oddsidemargin}{0.in}
\setlength{\evensidemargin}{0.in}

\setlength{\parindent}{0.em}
\setlength{\parskip}{0.75\baselineskip}

\def\boxwidth{\textwidth}
\def\doctitle{Cosmic Microwave Background $I$, $Q$, and $U$ Sky Generation}         % Document title defined here
\def\myname{Justin Lazear}      % Primary author defined here
\def\myemail{\texttt{jlazear@pha.jhu.edu}} % Primary author email
\def\piper{\textsc{Piper}}
\def\tablefont{\footnotesize}

% \renewcommand{\thefootnote}{\fnsymbol{footnote}}

%% \pagestyle{fancy}
%% \fancyhead{}
%% \fancyhead[LE,RO]{{\it {\large \thepage}}}
%% \fancyhead[RE]{{\it {\large \myname{}}}}
%% \fancyhead[LO]{{\it {\large \doctitle{}}}}
%% \fancyfoot{}

%%%%%%%%%%%%%%%%%%%%%%%%%%%%%%%%%%%%%%%%%%%%%%%%%%%%%%%%%%%%%%%%%%%%%%

% Define some math macros for convenience

% Note that most necessarily math macros are defined by the cool
% package, so this section is sparse.

\newcommand{\ve}[1]{\mathbf{#1}}
\newcommand{\Avg}[1]{\left< #1 \right>}
\newcommand{\pp}[1]{\left( #1 \right)}
\newcommand{\pb}[1]{\left[ #1 \right]}
\newcommand{\pc}[1]{\left{ #1 \right} }
\newcommand{\dd}[0]{\mathrm{d}}
\newcommand{\ClTT}[0]{C_\ell^{TT}}
\newcommand{\ClTE}[0]{C_\ell^{TE}}
\newcommand{\ClEE}[0]{C_\ell^{EE}}
\newcommand{\ClBB}[0]{C_\ell^{BB}}

%%%%%%%%%%%%%%%%%%%%%%%%%%%%%%%%%%%%%%%%%%%%%%%%%%%%%%%%%%%%%%%%%%%%%%

\title{\doctitle}
\author{\myname\
  \small \myemail}
\date{Feb 4, 2015}

\begin{document}
\maketitle

\section{Introduction}
\label{sec:intro}

We require a realization of the Cosmic Microwave Background (CMB) in both
intensity and polarization. We briefly discuss here the simplified methods
used to generate sky realizations of the CMB in $I$, $Q$, and $U$. The details
of the CMB sky are not important for this project, so more sophisticated
models are not required.

\section{CLASS}
\label{sec:class}

The Cosmic Linear Anisotropy Solving System
(CLASS)\footnote{\url{http://class-code.net}} software package is used to
generate power spectra ($\ClTT$, $\ClEE$, $\ClBB$, $\ClTE$). We use only a
small fraction of the power of CLASS, using primarily default parameters.
This is done because the things we would like to investigate here (e.g.
foreground removal, calibration sensitivity) are not sensitive to the details
of the CMB that is created, so there is no need to generate sophisticated
CMB skies. We note that the default parameters should show agreement with
the Planck 2013 parameters\cite{planck_collaboration_planck_2014}. The
parameters actually used are stored in
\texttt{data/lcdm\_planck\_2013\_parameters.ini}, which is also a valid input
file to CLASS. The resulting power spectra (both including and excluding
gravitational lensing) are also stored in the \texttt{data} directory.

\section{HEALPix and synfast}
\label{sec:healpix_and_synfast}

The power spectra are then read into Python and the \texttt{synfast} routine
of HEALPix\cite{gorski_healpix:_2005} is used to generate a realization of
the power spectrum in map space. In particular, the
\texttt{healpy}\footnote{\url{http://healpy.readthedocs.org}} interface to
HEALPix is used. We note that the \texttt{healpy.synfast} function requires
the power spectra to include $\ell = 0$ and $\ell = 1$ elements, even though
they are 0 (and thus typically excluded from Boltzmann codes like CLASS). We
extend the power spectar to include these elements (with value 0) before
passing them into \texttt{healpy.synfast}.

\bibliography{cmb}
\bibliographystyle{plain}

\end{document}